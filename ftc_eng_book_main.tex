%%%%%%%%%%%%%%%%%%%%%%%%%%%%%%%%%%%%%%%%%
% The Legrand Orange Book
% LaTeX Template
% Version 2.3 (8/8/17)
%
% This template has been downloaded from:
% http://www.LaTeXTemplates.com
%
% Original author:
% Mathias Legrand (legrand.mathias@gmail.com) with modifications by:
% Vel (vel@latextemplates.com)
%
% License:
% CC BY-NC-SA 3.0 (http://creativecommons.org/licenses/by-nc-sa/3.0/)
%
% Compiling this template:
% This template uses biber for its bibliography and makeindex for its index.
% When you first open the template, compile it from the command line with the 
% commands below to make sure your LaTeX distribution is configured correctly:
%
% 1) pdflatex main
% 2) makeindex main.idx -s StyleInd.ist
% 3) biber main
% 4) pdflatex main x 2
%
% After this, when you wish to update the bibliography/index use the appropriate
% command above and make sure to compile with pdflatex several times 
% afterwards to propagate your changes to the document.
%
% This template also uses a number of packages which may need to be
% updated to the newest versions for the template to compile. It is strongly
% recommended you update your LaTeX distribution if you have any
% compilation errors.
%
% Important note:
% Chapter heading images should have a 2:1 width:height ratio,
% e.g. 920px width and 460px height.
%
%%%%%%%%%%%%%%%%%%%%%%%%%%%%%%%%%%%%%%%%%

%----------------------------------------------------------------------------------------
%	PACKAGES AND OTHER DOCUMENT CONFIGURATIONS
%----------------------------------------------------------------------------------------

\documentclass[11pt,fleqn]{book} % Default font size and left-justified equations

%----------------------------------------------------------------------------------------

\input{structure} % Insert the commands.tex file which contains the majority of the structure behind the template
\usepackage{listings}
\usepackage{color}

%% from https://en.wikibooks.org/wiki/LaTeX/Source_Code_Listings
\definecolor{mygreen}{rgb}{0,0.6,0}
\definecolor{mygray}{rgb}{0.5,0.5,0.5}
\definecolor{mymauve}{rgb}{0.58,0,0.82}

\lstset{ %
	backgroundcolor=\color{white},   % choose the background color; you must add \usepackage{color} or \usepackage{xcolor}; should come as last argument
	basicstyle=\footnotesize,        % the size of the fonts that are used for the code
	breakatwhitespace=false,         % sets if automatic breaks should only happen at whitespace
	breaklines=true,                 % sets automatic line breaking
	captionpos=b,                    % sets the caption-position to bottom
	commentstyle=\color{mygreen},    % comment style
	deletekeywords={...},            % if you want to delete keywords from the given language
	escapeinside={\%*}{*)},          % if you want to add LaTeX within your code
	extendedchars=true,              % lets you use non-ASCII characters; for 8-bits encodings only, does not work with UTF-8
	frame=single,	                   % adds a frame around the code
	keepspaces=true,                 % keeps spaces in text, useful for keeping indentation of code (possibly needs columns=flexible)
	keywordstyle=\color{blue},       % keyword style
	language=Octave,                 % the language of the code
	morekeywords={*,...},            % if you want to add more keywords to the set
	numbers=left,                    % where to put the line-numbers; possible values are (none, left, right)
	numbersep=5pt,                   % how far the line-numbers are from the code
	numberstyle=\tiny\color{mygray}, % the style that is used for the line-numbers
	rulecolor=\color{black},         % if not set, the frame-color may be changed on line-breaks within not-black text (e.g. comments (green here))
	showspaces=false,                % show spaces everywhere adding particular underscores; it overrides 'showstringspaces'
	showstringspaces=false,          % underline spaces within strings only
	showtabs=false,                  % show tabs within strings adding particular underscores
	stepnumber=2,                    % the step between two line-numbers. If it's 1, each line will be numbered
	stringstyle=\color{mymauve},     % string literal style
	tabsize=2,	                   % sets default tabsize to 2 spaces
	title=\lstname                   % show the filename of files included with \lstinputlisting; also try caption instead of title
}

\begin{document}
%  Document EVERYTHING. 
%Include the time after you finish your build
%and all the way up to Competition, 
% as well as between Competitions. 
%If your Team plans on competing in multiple events for the season, what are you doing to improve your performance? 
%How are your outreach efforts? 
%How is your Team  investing its time in between the Competitions? 
%Do not stop using the Engineering Notebook once the robot is completed.  

%----------------------------------------------------------------------------------------
%	TITLE PAGE
%----------------------------------------------------------------------------------------

\begingroup
\thispagestyle{empty}
\begin{tikzpicture}[remember picture,overlay]
\node[inner sep=0pt] (background) at (current page.center) {\includegraphics[width=\paperwidth]{background}};
\draw (current page.center) node [fill=ocre!30!white,fill opacity=0.6,text opacity=1,inner sep=1cm]{\Huge\centering\bfseries\sffamily\parbox[c][][t]{\paperwidth}{\centering
%% The Team Number and Team Name must appear on the outside of the Engineering Notebook.
Team 0418 I'm a teapot  % Book title
\\[15pt]
{\Large 
%% Season year and challange
2017-2018 Relic Recovery % Subtitle
}\\[20pt] 
{\huge 
%% Meeting Location
Chateau de Chambord  % Author name
}}};
\end{tikzpicture}
\vfill
\endgroup
\input{book_sections/copyright}
\newpage
\thispagestyle{empty}
Attach a summary page to the front cover of the Engineering Notebook. 
%Team number
%Summary narrative about the Team, school, or organization 
%with bulleted highlights of the Team’s season. 
%list of pages in the Engineering Notebook that the Team would most like the Judges to consider.
% Summary Narrative Page should be used as a way to impress the Judges without the Team being in the room. 
%When crafting your summary narrative, make sure you highlight what makes your Team stand out. 
%Remember to keep this short, as you would want Judges to have more time diving into the pages of your notebook that you have tabbed for them. 
%----------------------------------------------------------------------------------------
%	TABLE OF CONTENTS
%----------------------------------------------------------------------------------------

%\usechapterimagefalse % If you don't want to include a chapter image, use this to toggle images off - it can be enabled later with \usechapterimagetrue

\chapterimage{chapter_head_1.pdf} % Table of contents heading image

\pagestyle{empty} % No headers

\tableofcontents % Print the table of contents itself

\cleardoublepage % Forces the first chapter to start on an odd page so it's on the right

\pagestyle{fancy} % Print headers again

%The Engineering Notebook must be divided into multiple sections, including:  
%An Engineering Section that includes the Robot design processes (required). 
\part{Engineering Section}
 
\chapter{Season Starter}
%Form borrowed from FTC 7043 Golden Gears & FTC 9774 Nano Ninjas
\begin{description}
	%Team Number/Name:  
	\item[Team Number/Name:] Team 0418 I'm a teapot 
	%Date of Meeting:  *   
	\item[Meeting Date:] Mon, 09/04/2018, 6:30 PM - 8:30 PM
	%Subcommittee:  *   
	\item[Subcommittee:] General Meeting
	% General Meeting  Business Plan  Design/Build  CAD  Programming  Marketing  %Safety  Drivers  Video  Social Media  Web  Outreach  Finance/Fundraising   
	
	%Attending:  *   
	\item[Personnel Present:] 
	Sample1, 
	%Sample2, 
	Sample3, 
	Sample5
	%   Name 1      Name 2      Name 3      Name 4      Name 5      Name 6   
	\item[Coaches:] 
	Sample1
	%  Coaches:  *   
	%   Coach 1      Coach 2   
	
	
	%Mentors/Guests:  
	%Objectives:  *   
	\item[Tasks This Meeting:] \
	\begin{itemize}
		\item Discuss Engineering Notebook\index{Engineering Notebook}
		\item Work on fundraisers\index{fundraiser}
		\item Research robot building
		\item Outreach planning
	\end{itemize}
	%Reflections:  *   
	\item[Reflections:] \
	
	At the meeting we discussed 
	
	%Action Items:  
	\textit{Engineering Notebook and Outreach} 
	\begin{itemize}  
		\item Engineering log must be concise, and worked on be all team members.
	\end{itemize}
	
	%\includegraphics[width=.9\textwidth]{Images/Draw1.PNG}\\
	
	%\includegraphics[width=.9\textwidth]{Images/Draw2.PNG}\\
	%Filled out by:  *  
	\item [Submitted By sample3]\
\end{description}

\lstinputlisting[language=java]{code/ConceptRampMotorSpeed.java}
% Ask yourself questions like: What worked, what didn’t? 
%Do not be afraid to include your failures. 
%When something didn’t work, how did you problem-solve? 
%What modifications are you planning on making?  
%How has your robot changed over the course of the season? 
%The documentation should be detailed enough that somebody could look at your notebook and be able to build your robot from it.
% This documentation should include: 
%  Sketches 
% Discussions and Team meetings 
% Design evolution 
%  Software development 
%processes, obstacles 
%Start a fresh page in the Engineering Section at every meeting. The date, and start/stop times recorded when starting a new page.
%Include highlights from the meeting. What was talked about? What did you do? What was 
%planned and/or accomplished during your time together? 
%Highlight unique design or playing strategy.
%All designs and changes with detailed notes, calculations,  and sketches if possible.
%explain the underlying science, math, and strategies your Team is using as well as why you are doing what you are doing. 
%Pictures or sketches of the Robot designs,
%electrical wiring diagrams 
%software development describe your code.  
%Entries should be made by every Team member, initialed, and dated.
%showing that your Team has multiple members sharing the responsibility showcases sustainability of your Team.

%A Team Section that includes information about the Team (required).
\part{Team Section}
\chapter{Team Members}
\section {Student1}
\index{Student1} 
%\begin{wrapfigure}{R}{0.15\textwidth}
%\centering
%\includegraphics[width=0.15\textwidth]{Images/Student1.JPG}
%\end{wrapfigure}
Student1 Introducing each Team member with a Pictures and  brief  biography of their name, age (or school year),\\ role on the Team,\\ interests, and reasons for joining an FIRST Tech Challenge Team. \\
\section {Student1}
\index{Student1} 
%\begin{wrapfigure}{R}{0.15\textwidth}
%\centering
%\includegraphics[width=0.15\textwidth]{Images/Student1.JPG}
%\end{wrapfigure}
Student1 Introducing each Team member with a Pictures and  brief  biography of their name, age (or school year),\\ role on the Team,\\ interests, and reasons for joining an FIRST Tech Challenge Team. \\ % just to prove more than 1 file can be added
\chapter{Team Mentors \& Coach}
\section {Sample}
\index{Sample} 
%\begin{wrapfigure}{R}{0.15\textwidth}
%\centering
%\includegraphics[width=0.15\textwidth]{Images/Sample.jpg}
%\end{wrapfigure}
Coach Sample Introducing each Mentor with a brief  biography of their name, age (or school year), role on the Team, interests, and reasons for joining an FIRST Tech Challenge Team.\\
Discuss and show Team activities that are done throughout the Team’s season.
%Introducing each Team member with a Pictures and  brief  biography of their name, age (or school year), role on the Team, interests, and reasons for joining an FIRST Tech Challenge Team. 
%Introducing each Mentor with a brief  biography of their name, age (or school year), role on the Team, interests, and reasons for joining an FIRST Tech Challenge Team.
% Discuss and show Team activities that are done throughout the Team’s season.
% Each Team member’s thoughts throughout the journey for the entire season 

%Outreach activities (required). 
\part{Outreach Section}
%These can include what your Team outreach efforts include, 
%Team building activities, or more. 

%A Business plan (not required)
\part{Business Plan Section}
% How do you plan to fund your way to each event? 
% Direction the Team wants to take. 
% Outlining Team Goals. 
% Type of outreach Team wants to focus on. 
% Creating a Team budget. 
% Fundraising needs. 
% Seeking out Sponsors.
%o  How much money does your Team need to get started and for what? (ie: a budget) 
%o  How much money does your Team need to sustain it for multiple years? 
%o  How much money does your Team plan to raise through fundraising efforts? 
%o  How much money does your Team anticipate raising through grant opportunities? 
%o  How much money/in-kind donations does your Team plan to raise through sponsorship? 
%o  How will you promote a business/organization if you receive a sponsorship? (Logo on Team t-shirts and flare, etc.) 
%Make sure to identify what must be purchased, such as tools and major components, not only simple categories like parts, travel lodging, etc.  
%It should include the budget, the fundraising plan, 
%contingencies if the Team does not reach their fund raising plan, and what they will do if they have excess.

%A Strategic plan (not required)
\part{Strategic Plan Section}
%Your Strategic Plan may also include milestones. These may they be short term (i.e. competing in at least two events this season) to one year plan (i.e. making it to a Regional Championship and fundraising the costs) or even further (i.e. plans to create and foster additional FIRST Tech Challenge Teams).

%A Sustainability plan (not required)  
\part{Sustainability Plan Section}
%When considering including a Sustainability Plan, ask yourself: 
% How does your Team plan on continuing past this season? 
% Are you training younger Team members in key roles as members graduate out? 
% Are there plans for recruiting new members? What are they? 
% Does your Team have plans on recruiting additional Mentors? 
% How are you fundraising the costs of next season? Registration? New parts? 
% How are you and your Team leaving a legacy that will last beyond just this season? 
\part{End}
%----------------------------------------------------------------------------------------
%	BIBLIOGRAPHY
%----------------------------------------------------------------------------------------

\chapter*{Bibliography}
\addcontentsline{toc}{chapter}{\textcolor{ocre}{Bibliography}}

%------------------------------------------------

\section*{Articles}
\addcontentsline{toc}{section}{Articles}
%\printbibliography[heading=bibempty,type=article]

%------------------------------------------------

\section*{Books}
\addcontentsline{toc}{section}{Books}
%\printbibliography[heading=bibempty,type=book]

%----------------------------------------------------------------------------------------
%	INDEX
%----------------------------------------------------------------------------------------
%NOT Used
%\cleardoublepage
%\phantomsection
%\setlength{\columnsep}{0.75cm}
%\addcontentsline{toc}{chapter}{\textcolor{ocre}{Index}}
%\printindex
%----------------------------------------------------------------------------------------
\end{document}
